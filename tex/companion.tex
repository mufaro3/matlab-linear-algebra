\documentclass[12pt]{article}
\usepackage{parskip, amsmath, physics, multicol}

\newcommand{\mat}[1]{\mathbf{#1}}
\newcommand{\exercise}[1]{\uppercase{\textbf{EXERCISE #1}}\label{#1}}

\begin{document}
\exercise{2(g)}

Given
\begin{equation*}
  \mat{M} = \begin{pmatrix}
    1 & 2 & 0 \\
    0 & 0 & 3 \\
    0 & 1 & 0
  \end{pmatrix},
\end{equation*}
we can produce $\mat{M} \to \mat{I}_3$ via:
\begin{enumerate}
\item Divide row $3$ by 3.
\item Add multiples of -2 from row 2 to row 1.
\item Swap rows 2 and 3\footnote{This was originally wrong and placed first. You need to swap last because matrix multiplication is right-associative, so the rightmost operation occurs first, not last. In other words, a ``correct'' way to approach this would be to do the entire process in reverse when multiplying.}.
\end{enumerate}
This effect can be produced by the following elementary matrices. 

The swap matrix can be created from emulation with $I_3$:
\begin{equation*}
\mat{E}_{swap} = 
\begin{pmatrix}
 1 & 0 & 0 \\
 0 & 0 & 1 \\
 0 & 1 & 0
\end{pmatrix},
\end{equation*}
similarly, the division matrix would be
\begin{equation*}
\mat{E}_{div} =
\begin{pmatrix}
 1 & 0 & 0 \\
 0 & 1 & 0 \\
 0 & 0 & \frac{1}{3}
\end{pmatrix},
\end{equation*}
and lastly, the additive matrix would be
\begin{equation*}
\mat{E}_{add} =
\begin{pmatrix}
1 & -2 & 0 \\
0 & 1  & 0 \\
0 & 0  & 1
\end{pmatrix}.
\end{equation*}
Altogether, the following multiplication should hold true:
\begin{equation*}
\mat{E}_{div} \mat{E}_{add} \mat{E}_{swap} \mat{M} = \mat{I}_{3}
\end{equation*}
\newpage

\exercise{2(h)}

Given
\begin{align*}
x + 2y &= 4 \\
3z &= 6 \\
y &= 8,
\end{align*}
we can produce the following coefficient matrix $\mat{A}$ and constant vector $\mat{B}$:
\begin{equation*}
\mat{A} = 
\begin{pmatrix} 
1 & 2 & 0 \\ 
0 & 0 & 3 \\ 
0 & 1 & 0 
\end{pmatrix} 
\hspace{0.10\linewidth} 
\mat{B} = 
\begin{pmatrix} 
4 \\ 6 \\ 8
\end{pmatrix},
\end{equation*}
and given that $\mat{A} = \mat{M},$ from exercise 2(g) the solutions can be calculated by performing the RREF calculations on $\mat{B}$ to produce solution vector $\mat{S}$:
\begin{equation*}
\mat{E}_{div} \mat{E}_{add} \mat{E}_{swap} \mat{B} = \mat{S},
\end{equation*}
yielding
\begin{equation*}
\mat{S} = \begin{pmatrix} -12 \\ 8 \\ 2 \end{pmatrix},
\end{equation*}
showing that $x=-12, y = 8, z=2$.

\exercise{2(i)}

This isn't solvable because the best RREF form is
\begin{equation*}
\mat{[M | B]} = 
\begin{pmatrix}
1 & 0 & -6 & 4 \\
0 & 1 &  3 & 6 \\
0 & 0 &  0 & 6 
\end{pmatrix}
\end{equation*}
or, in other words, equations (2) and (3) contradict each other.

\exercise{2(j)}

The initial state of the system is
\begin{equation*}
\begin{pmatrix}
1 & 2 & 0 & 4 \\
0 & 1 & 3 & 6 \\
0 & 2 & 6 & 12
\end{pmatrix},
\end{equation*}
and the RREF of this system becomes
\begin{equation*}
\begin{pmatrix}
1 & 0 & -6 & -8 \\
0 & 1 & 3 & 6 \\
0 & 0 & 0 & 0
\end{pmatrix}.
\end{equation*}
Again, this system is not solvable as equations 2 and 3 are equivalent, resulting in this system not containing enough information to be adequately solved, but the reduced echelon form of the coefficient matrix is the same as in exercise 2(i).

\end{document}
